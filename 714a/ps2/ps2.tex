% $•$% !TeX root = macro_PS6_case.tex

\documentclass[]{article}
\usepackage{amssymb}
\usepackage{amsmath}
\usepackage{enumitem}
\usepackage{bm}
\usepackage{cases}
\usepackage{changepage}\usepackage{amsmath}
\usepackage{amsfonts}
\usepackage{centernot}
\usepackage{graphicx}
\usepackage{float}
\usepackage{hyperref} % to restart section numbers for different parts
\usepackage{physics} % to get partial derivatives 
\usepackage{xcolor}
\usepackage[a4paper, total={6in, 8in}]{geometry}
\usepackage{listings} % to input code
       \lstset{backgroundcolor = \color{white},
               % frame = single,
               %keywordstyle=\color{blue}
               }

\setlength{\parindent}{0em} % no automatic indents

%opening
\title{Econ 714 Quarter 1: Problem set 2 }
\author{Emily Case}

% command for blank footnote 
\newcommand\blfootnote[1]{%
	\begingroup
	\renewcommand\thefootnote{}\footnote{#1}%
	\addtocounter{footnote}{-1}%
	\endgroup
}

\newcount\colveccount
\newcommand*\colvec[1]{
	\global\colveccount#1
	\begin{pmatrix}
		\colvecnext
	}
	\def\colvecnext#1{
		#1
		\global\advance\colveccount-1
		\ifnum\colveccount>0
		\\
		\expandafter\colvecnext
		\else
	\end{pmatrix}
	\fi
}

\newcommand{\R}{\mathbb{R}}
\newcommand{\co}[2]{c_{#1}^{#2}} % makes subscript and superscript easier for consumption
\newcommand{\spa}{\text{ }}
\newcommand{\lnl}{\ell_n} % log likelihood function
\newcommand{\bxn}{\bar{X}_n}
\newcommand{\lag}{\mathcal{L}}
\newcommand{\sumin}{\sum\limits_{i=1}^n} % generic sumation 1 to n
\newcommand{\sumti}{\sum\limits_{t=0}^\infty} % sum - infinite horizon
\newcommand{\pin}{\Pi_{i=1}^n}
\newcommand{\argmax}{\text{arg}\max}
\newcommand{\fix} [1] {\textbf{\textcolor{blue}{#1}}} % use this to more clearly note in problem set pdf where things need to be changed. 
% end preamble

\renewcommand*{\thesection}{\arabic{section}}
\newcommand\numberthis{\addtocounter{equation}{1}\tag{\theequation}} % for equation numbering within align*

\begin{document}
	
	\maketitle
	
	\blfootnote{I worked on this Problem set with Sarah Bass, Michael Nattinger, Alex von Hafften, and Danny Edgel.} 


%%%%%%%%%%%%%%%%%%%%%%%%%%%%%%%%%%%%%%%%%%%%%%%%%%%%%%%%%%

\section*{Problem 1}
Consider a growth model with preferences $\sumti \beta^t\log C_t$, production function $Y_t=AK_t^\alpha$,the capital law of motion $K_{t+1}=K_{t}^{1-\delta}I_t^\delta$, and the resource constraint $Y_t=C_t+I_t$.

%%% num. 1 %%%
\subsection*{(1) Write down the social planner’s problem and derive the Euler equation. Provide the intuition to this optimality condition using the perturbation argument.} 

% social planner problem %
\begin{align*}
\max_{\{C_t\}_{t=0}^\infty} & \sumti \beta^t \log C_t 
\\
\text{s.t.}\; & K_{t+1}
 = K_t^{1-\delta}(AK_t^\alpha-C_t)^\delta
\end{align*}


Set up the lagrangian:
\[ \mathcal{L} = 
\sumti \beta^t \log C_t -\lambda_t 
[K_t^{1-\delta}(AK_t^\alpha -C_t)^\delta -K_{t+1} ]\]

FOC[$C_t$]:
\[\frac{\beta^t}{C_t} = -\lambda_t K_t^{1-\delta}(-1)\delta(AK_t^\alpha-C_t)^{\delta-1}
\]
which also gives us 
\[ \lambda_t = \frac{-\beta^t}{\delta C_t K_t^{1-\delta}(AK_t^\alpha-C_t)^{\delta-1}}\]

FOC[$K_{t+1}$]:
\[\lambda_t = \lambda_{t+1} [
k_{t+1}^{-\delta}(1-\delta)(AK_{t+1}^\alpha-C_{t+1})^\delta + 
\delta K_{t+1}^{1-\delta}(AK_t^\alpha -C_t)^{\delta-1}\alpha AK_{t+1}^{\alpha-1}]
\]


Combining, and replacing the $I_t$'s, we get the Euler Equation: 
\begin{align*}
\frac{-\beta^t}{\delta C_t K_t^{1-\delta}I_t^{\delta-1}} 
& = \frac{-\beta^{t+1}}{\delta C_{t+1} K_{t+1}^{1-\delta}I_{t+1}^{\delta-1}} 
\left[ k_{t+1}^{-\delta}(1-\delta)I_{t+1}^\delta + 
\delta K_{t+1}^{1-\delta}I_t^{\delta-1}\alpha AK_{t+1}^{\alpha-1}\right] 
\\
\frac{1}{C_tK_t^{1-\delta}I_t^{\delta-1}}
& = \frac{\beta}{C_{t+1}K_{t+1}}
\left[ (1-\delta)I_{t+1}+\delta\alpha A K_{t+1}^\alpha \right]
\numberthis \label{euler}
\end{align*}

\fix{need to do perturbation argument} 
% end num. 1 %


%%% num. 2 %%%  
\subsection*{(2) Derive the system of equations that pins down the steady state of the model.}

Using the Euler Equation \eqref{euler} and the capital law of motion, 
\begin{equation} \label{LOM}
K_{t+1}=K_{t}^{1-\delta}I_t^\delta 
\end{equation}
now with $C_t = C_{t+1} = \bar{C}$, $K_t=K_{t+1} = \bar{K}$, and $I_t = I_{t+1} = \bar{I}$\footnote{Where $\bar{I} = A\bar{K}^\alpha -\bar{C}$}, we get:

%% steady state %%
\begin{align}
\eqref{euler} & \Rightarrow 1 = \beta \bar{K}^{-\alpha}\bar{I}^{\delta-1}[\delta\alpha A \bar{K}^\alpha + (1-\delta)\bar{I} ]
\label {eulerSS}\\
\eqref{LOM} & \Rightarrow 1 = \bar{K}^{-\delta}\bar{I}^\delta \label{LOMSS}
\end{align}
which define the steady state. Notice also that \eqref{LOMSS} implies $\bar{K} = \bar{I}$, which will be useful. 
% end num. 2 %


%%% num. 3 %%%
\subsection*{(3) Log-linearize the equilibrium conditions around the steady state.} 
First log-linearize $I_t = AK_t^\alpha -C_t$:
\[i_t = \frac{\alpha A\bar{K}^\alpha}{\bar{I}}k_t -\frac{\bar{C}}{\bar{I}}c_t
= \alpha A\bar{K}^{\alpha-1}k_t -\frac{\bar{C}}{\bar{I}}c_t\]

Now I log-linearize equation \eqref{LOM}:
\begin{align*}
\bar{K}(1+k_{t+1}) & = \bar{K}^{1-\delta}[1+(1-\delta)k_t]\bar{I}^\delta (1+\delta i_t)
\\
k_{t+1} & = (1-\delta)k_t+\delta i_t 
\\
& = (1-\delta)k_t+\delta \left(\alpha A\bar{K}^{\alpha-1}k_t -\frac{\bar{C}}{\bar{I}}c_t\right)
\\
& = k_t\left(1-\delta +\delta\alpha A\bar{K}^{\alpha-1}\right) -\delta\frac{\bar{C}}{\bar{I}}c_t
\end{align*} 
\fix{check this against someone's}

And finally, equation \eqref{euler}:


%%% num. 4 %%%
\subsection*{(4) Write down a dynamic system with one state variable and one control variable. Use the Blanchard-Kahn method to solve this system for a saddle path.}


%%% num. 5 %%%
\subsection*{(5) Show that the obtained solution is not just locally accurate, but is in fact the exact solution to the planner’s problem.}


%%% num. 6 %%%
\subsection*{(6) Generalize the (global) solution to the case of stochastic productivity shocks $A_t$.}


%%% num. 7 %%%
\subsection*{(7) The analytical tractability of the model is due to special functional form assumptions,which however, have strong economic implications. What is special about consumption behavior in this model? Provide economic intuition.}


%%% num. 8 %%%
\subsection*{(8) \textit{Bonus task:} can you introduce labor into preferences and production function without compromising the analytical tractability of the model?}
\fix{I will probably not do this} 



\end{document}
