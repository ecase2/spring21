% $•$% !TeX root = macro_PS6_case.tex

\documentclass[]{article}
\usepackage{amssymb}
\usepackage{amsmath}
\usepackage{enumitem}
\usepackage{bm}
\usepackage{cases}
\usepackage{changepage}\usepackage{amsmath}
\usepackage{amsfonts}
\usepackage{centernot}
\usepackage{graphicx}
\usepackage{float}
\usepackage{hyperref} % to restart section numbers for different parts
\usepackage{physics} % to get partial derivatives 
\usepackage{xcolor}
\usepackage[a4paper, total={6in, 8in}]{geometry}
\usepackage{listings} % to input code
       \lstset{backgroundcolor = \color{white},
               % frame = single,
               %keywordstyle=\color{blue}
               }

%opening
\title{Econ 714 Quarter 1: Problem set 1 }
\author{Emily Case}

% command for blank footnote 
\newcommand\blfootnote[1]{%
	\begingroup
	\renewcommand\thefootnote{}\footnote{#1}%
	\addtocounter{footnote}{-1}%
	\endgroup
}

\newcount\colveccount
\newcommand*\colvec[1]{
	\global\colveccount#1
	\begin{pmatrix}
		\colvecnext
	}
	\def\colvecnext#1{
		#1
		\global\advance\colveccount-1
		\ifnum\colveccount>0
		\\
		\expandafter\colvecnext
		\else
	\end{pmatrix}
	\fi
}

\newcommand{\R}{\mathbb{R}}
\newcommand{\co}[2]{c_{#1}^{#2}} % makes subscript and superscript easier for consumption
\newcommand{\spa}{\text{ }}
\newcommand{\lnl}{\ell_n} % log likelihood function
\newcommand{\bxn}{\bar{X}_n}
\newcommand{\lag}{\mathcal{L}}
\newcommand{\sumin}{\sum\limits_{i=1}^n} % generic sumation 1 to n
\newcommand{\sumti}{\sum\limits_{t=1}^\infty} % sum - infinite horizon
\newcommand{\pin}{\Pi_{i=1}^n}
\newcommand{\argmax}{\text{arg}\max}
\newcommand{\fix} [1] {\textbf{\textcolor{blue}{#1}}} % use this to more clearly note in problem set pdf where things need to be changed. 
% end preamble

\renewcommand*{\thesection}{\arabic{section}}

\begin{document}
	
	\maketitle
	
	\blfootnote{I worked on this Problem set with Sarah Bass, Michael Nattinger, Alex von Hafften, and Danny Edgel.} 


%%%%%%%%%%%%%%%%%%%%%%%%%%%%%%%%%%%%%%%%%%%%%%%%%%%%%%%%%%
Consider a neoclassical growth model with preferences $\sum_{t=0}^\infty \beta^t U(C_t)$, production technology $F(K_t)$,  and the initial capital endowment $K_0$.  Both $U(\cdot)$ and $F(\cdot)$ are strictly increasing, strictly concave and satisfy standard Inada conditions.  The capital law of motion is \[K_t+1= (1-\delta)K_t+I_t-D_t\] where $D_t$ is a natural disaster shock that destroys a fixed amount of the accumulated capital.

\section{Write down the social planner’s problem and derive the intertemporal optimality condition (the Euler equation).}


\section{Given  the  steady-state  value  of $D\ge 0$,  write  down  the  system  of  equations  that determines  the  values  of  capital $\bar{K}(D)$ and  consumption $\bar{C}(D)$  in  the  steady  state. Draw  a  phase  diagram  with  capital  in  the  horizontal  axis  and  consumption  in  the vertical  axis,  show  the  steady  states,  draw  the  arrows  representing the direction of change, and the saddle path.}

\section{The  scientists  forecast  an  earthquake $T$ periods  from  now  that  will  destroy $D >0$ units of capital.  Assuming that economy starts from a steady state with $D= 0$, draw a phase diagram that shows the optimal transition path.  Make two separate graphs showing the evolution of capital and consumption in time.}

\section{Assume that $U(C) = \frac{C^{1-\sigma}-1}{1-\sigma} $ and $F(K)= K^\alpha$ and the values of parameters are $\sigma= 1,\;\alpha= 1/3,\;\beta= 0.99^{1/12}$ (monthly model), $\delta = 0.01,\; T= 12,\;D= 1$.  Using a shooting algorithm,  solve  numerically  for  the  optimal  transition  path  and  plot  dynamics  of consumption and capital.}


\end{document}
