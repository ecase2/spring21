% $•$% !TeX root = macro_PS6_case.tex

\documentclass[]{article}
\usepackage{amssymb}
\usepackage{amsmath}
\usepackage{enumitem}
\usepackage{bm}
\usepackage{cases}
\usepackage{changepage}\usepackage{amsmath}
\usepackage{amsfonts}
\usepackage{centernot}
\usepackage{graphicx}
\usepackage{float}
\usepackage{hyperref} % to restart section numbers for different parts
\usepackage{physics} % to get partial derivatives 
\usepackage{xcolor}
\usepackage[a4paper, total={6in, 8in}]{geometry}
\usepackage{listings} % to input code
       \lstset{backgroundcolor = \color{white},
               % frame = single,
               %keywordstyle=\color{blue}
               }

%opening
\title{Econ 714 Quarter 1: Problem set 1 }
\author{Emily Case}

% command for blank footnote 
\newcommand\blfootnote[1]{%
	\begingroup
	\renewcommand\thefootnote{}\footnote{#1}%
	\addtocounter{footnote}{-1}%
	\endgroup
}

\newcount\colveccount
\newcommand*\colvec[1]{
	\global\colveccount#1
	\begin{pmatrix}
		\colvecnext
	}
	\def\colvecnext#1{
		#1
		\global\advance\colveccount-1
		\ifnum\colveccount>0
		\\
		\expandafter\colvecnext
		\else
	\end{pmatrix}
	\fi
}

\newcommand{\R}{\mathbb{R}}
\newcommand{\co}[2]{c_{#1}^{#2}} % makes subscript and superscript easier for consumption
\newcommand{\spa}{\text{ }}
\newcommand{\lnl}{\ell_n} % log likelihood function
\newcommand{\bxn}{\bar{X}_n}
\newcommand{\lag}{\mathcal{L}}
\newcommand{\sumin}{\sum\limits_{i=1}^n} % generic sumation 1 to n
\newcommand{\sumti}{\sum\limits_{t=1}^\infty} % sum - infinite horizon
\newcommand{\pin}{\Pi_{i=1}^n}
\newcommand{\argmax}{\text{arg}\max}
\newcommand{\fix} [1] {\textbf{\textcolor{blue}{#1}}} % use this to more clearly note in problem set pdf where things need to be changed. 
% end preamble

\renewcommand*{\thesection}{\arabic{section}}

\begin{document}
	
	\maketitle
	
	\blfootnote{I worked on this Problem set with Sarah Bass, Michael Nattinger, Alex von Hafften, and Danny Edgel.} 


%%%%%%%%%%%%%%%%%%%%%%%%%%%%%%%%%%%%%%%%%%%%%%%%%%%%%%%%%%
Consider a neoclassical growth model with preferences $\sum_{t=0}^\infty \beta^t U(C_t)$, production technology $F(K_t)$,  and the initial capital endowment $K_0$.  Both $U(\cdot)$ and $F(\cdot)$ are strictly increasing, strictly concave and satisfy standard Inada conditions.  The capital law of motion is \[K_{t+1}= (1-\delta)K_t+I_t-D_t\] where $D_t$ is a natural disaster shock that destroys a fixed amount of the accumulated capital.

\section{Write down the social planner’s problem and derive the intertemporal optimality condition (the Euler equation).}
Note that $F(K_t) = C_t + I_t$. Then the social planner's problem is 
\begin{align*}
\max\limits_{C_t} & \sum_{t=0}^\infty \beta^t U(C_t)  \\
\text{s.t.}\; & F(K_t)  = C_t + K_{t+1} -(1-\delta)K_t +D_t 
    % the output of a period has to cover the consumption of that period, along with the capital bought for next period, not including the disaster shock. 
\end{align*}
Now we can set up the lagrangian:
\[\sum_{t=0}^\infty \beta^t \left[ U(C_t)+ \lambda_t (-F(K_t) + C_t + K_{t+1} -(1-\delta)K_t +D_t )\right]\]
FOC with respect to $C_t$:
\[U'(C_t) + \lambda^t = 0 \]
FOC with respect to $K_{t}$:
\[ \beta^{t+1} \lambda_{t+1} = \beta^t\lambda_t(1-\delta + F'(K_t)) 
\]
Simplifying and combining FOC, we can get the Euler Equation:
\[\beta \lambda_{t+1} = \lambda_t (1-\delta + F'(K_t)) \]
\[\beta U'(C_{t+1}) = U'(C_t) (1-\delta + F'(K_t)) \]
\[\beta\frac{U'(C_{t+1})}{U'(C_t)} = 1-\delta + F'(K_t) \]

\section{Given  the  steady-state  value  of $D\ge 0$,  write  down  the  system  of  equations  that determines  the  values  of  capital $\bar{K}(D)$ and  consumption $\bar{C}(D)$  in  the  steady  state. Draw  a  phase  diagram  with  capital  in  the  horizontal  axis  and  consumption  in  the vertical  axis,  show  the  steady  states,  draw  the  arrows  representing the direction of change, and the saddle path.}
Let $K_t = K_{t+1}=\bar{K}$ and $C_t= C_{t+1}=\bar{C}$, and the Euler equation becomes:
\[\beta = 1-\delta + F'(\bar{K}) \]
Also, the resource constraint becomes:
\[ F(\bar{K})  = \bar{C} + \delta\bar{K} +D \] 
\fix{this absolutely does not seem right}

\section{The  scientists  forecast  an  earthquake $T$ periods  from  now  that  will  destroy $D >0$ units of capital.  Assuming that economy starts from a steady state with $D= 0$, draw a phase diagram that shows the optimal transition path.  Make two separate graphs showing the evolution of capital and consumption in time.}
\fix{drawn on ipad but want to fix it so that shock puts us on the saddle path below new SS, so that they increase again. However the way I have it drawn maybe indicates that if you don't drop consumption low enough and save enough, you won't have great insurance for earthquake? ask about this...} \\
\\
\fix{could also plot on matlab using question 4's values? if i have time} 

\section{Assume that $U(C) = \frac{C^{1-\sigma}-1}{1-\sigma} $ and $F(K)= K^\alpha$ and the values of parameters are $\sigma= 1,\;\alpha= 1/3,\;\beta= 0.99^{1/12}$ (monthly model), $\delta = 0.01,\; T= 12,\;D= 1$.  Using a shooting algorithm,  solve  numerically  for  the  optimal  transition  path  and  plot  dynamics  of consumption and capital.}
Updated Euler Equation:
\begin{align*}
\beta\frac{C_{t+1}^{-\sigma}}{C_t^{-\sigma}}  & = 1-\delta + \alpha K_t^{1-\alpha} \\
\left(\frac{C_{t+1}}{C_t}\right)^{-\sigma}
& = \frac{1}{\beta} \left( 1-\delta + \alpha K_t^{1-\alpha} \right) \\
C_{t+1} & = C_t\beta^{1/\sigma} (1-\delta+\alpha K_t^{1-\alpha})^{-1/\sigma}
\end{align*}
Updated resource constraint:
\begin{align*}
K_t^\alpha = C_t + K_{t+1} -(1-\delta)K_t +D_t
\end{align*}
Notes:
\begin{itemize}
\item SS before shock: K = 170.57 and C = 3.84
\end{itemize}

\fix{\emph{General idea for code:}\\Use a consumption grid instead of a capital grid. initialize that and define the parameters. Find the steady state values, without the shock value first, because that steady state is where it will start at.\\\\Now use the shooting method to find the trajectory for this SS (no shock yet). Given a K, what would the optimal consumption be? do this for all K in order to find the saddle path trajectory. \\\\ in general very stuck here.... have half tried my own method half tried to learn Danny's...
}

\end{document}
