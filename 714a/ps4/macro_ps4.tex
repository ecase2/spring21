% $•$% !TeX root = macro_PS6_case.tex

\documentclass[12pt,oneside,reqno]{amsart}
    % \documentclass[]{article}
\usepackage{amssymb}
\usepackage{amsmath}
\usepackage{enumitem}
\usepackage{bm}
\usepackage{cases}
\usepackage{changepage}
\usepackage{amsfonts}
\usepackage{centernot}
\usepackage{graphicx}
\usepackage{float}
\usepackage{hyperref} % to restart section numbers for different parts
\usepackage{physics} % to get partial derivatives 
\usepackage{xcolor}
\usepackage[a4paper, total={6in, 8in}]{geometry}
\usepackage{listings} % to input code
       \lstset{backgroundcolor = \color{white},
               % frame = single,
               %keywordstyle=\color{blue}
               }
\usepackage{etoolbox} 
\patchcmd{\section}{\centering}{}{}{}
    % made section title not centered 
\let\flushleftsection\section% Copy updated non-centered definition of \section
\newcommand{\sectionscenter}{\let\section\centeredsection}% Switch to centered \section
\newcommand{\sectionsleft}{\let\section\flushleftsection}% Switch to flush left \section

\setlength{\parindent}{0em} % no automatic indents

%\usepackage{titlesec}
  %  \titleformat{\section}
  %      { \bfseries}{\thesection}{1em}{}

%opening
\title{Econ 714 Quarter 1: Problem set 4 }
\author{Emily Case}

% command for blank footnote 
\newcommand\blfootnote[1]{%
	\begingroup
	\renewcommand\thefootnote{}\footnote{#1}%
	\addtocounter{footnote}{-1}%
	\endgroup
}

\newcount\colveccount
\newcommand*\colvec[1]{
	\global\colveccount#1
	\begin{pmatrix}
		\colvecnext
	}
	\def\colvecnext#1{
		#1
		\global\advance\colveccount-1
		\ifnum\colveccount>0
		\\
		\expandafter\colvecnext
		\else
	\end{pmatrix}
	\fi
}

\newcommand{\R}{\mathbb{R}}
\newcommand{\E}{\mathbb{E}}
\newcommand{\co}[2]{c_{#1}^{#2}} % makes subscript and superscript easier for consumption
\newcommand{\spa}{\text{ }}
\newcommand{\lnl}{\ell_n} % log likelihood function
\newcommand{\bxn}{\bar{X}_n}
\newcommand{\lag}{\mathcal{L}}
\newcommand{\sumin}{\sum\limits_{i=1}^n} % generic sumation 1 to n
\newcommand{\sumti}{\sum\limits_{t=0}^\infty} % sum - infinite horizon
\newcommand{\pin}{\Pi_{i=1}^n}
\newcommand{\argmax}{\text{arg}\max}
\newcommand{\fix} [1] {\textbf{\textcolor{blue}{#1}}} % use this to more clearly note in problem set pdf where things need to be changed. 
% end preamble

\renewcommand*{\thesection}{\arabic{section}}
\newcommand\numberthis{\addtocounter{equation}{1}\tag{\theequation}} % for equation numbering within align*

\begin{document}
	
	\maketitle
	
	\blfootnote{I worked on this Problem set with Sarah Bass, Michael Nattinger, Alex von Hafften, Hanna Han, and Danny Edgel.} 


%%%%%%%%%%%%%%%%%%%%%%%%%%%%%%%%%%%%%%%%%%%%%%%%%%%%%%%%%%

% commands specific to this problem set % 
\newcommand{\sumi}{\sum_{i=1}^{N_k}}
\newcommand{\ik}{_{ik}}

\section*{Oligopolistic competition (Atkeson and Burstein, AER 2008)}
Consider a static model with a continuum of sectors $k\in[0,1]$ and $i= 1,...,N_k$ firms in sector $k$, each producing a unique variety of the good.  Households supply inelastically one unit of labor and have nested-CES preferences:
\begin{align*}
& C = \left( \int C_k^{\frac{\rho-1}{\rho}}dk\right)^{\frac{\rho}{\rho-1}},
& C_k = \left( \sum_{i=1}^{N_k} C_{ik}^{\frac{\theta-1}{\theta}}\right)^{\frac{\theta}{\theta-1}},
& \theta > \rho \ge 1.&
\end{align*}
For firm i in sector k: $ Y_{ik}= A_{ik}L_{ik}$ 



%%% 1 %%% 
\section{Solve household cost minimization problem for the optimal demand $C_{ik}$, the sectoral price index $P_k$, and the aggregate price index $P$ as functions of producers’ prices.}
\begin{align*}
& \min_{C_{ik}} \int \sumi P\ik C\ik dk \\
& s.t.\;\;\; 
C = \left( \int C_k^{\frac{\rho-1}{\rho}}dk\right)^{\frac{\rho}{\rho-1}},\;\;\; 
& C_k = \left( \sum_{i=1}^{N_k} C_{ik}^{\frac{\theta-1}{\theta}}\right)^{\frac{\theta}{\theta-1}}
\end{align*}

Construct a lagrangian:

\begin{align*}
\lag & = \int \sumi P\ik C\ik dk 
         - P \left[\left( C_k^\frac{{\rho-1}{\rho}} dk
         \right)
         ^{\frac{\rho}{\rho-1}} - C \right]
         + \int P_k \left[C_k - \left(\sumi C\ik 
         ^{\frac{\theta-1}{\theta}}
         \right)^{\frac{\theta
         }{\theta-1}}\right] dk
\end{align*}


\pagebreak
FOC $[C\ik]$:

\begin{align*}
P\ik & = P_k \left(\sumi C\ik^{\frac{\theta-1}{\theta}} \right)^{\frac{1}{\theta-1}} C\ik^{\frac{-1}{\theta}}
\\
\Rightarrow
C\ik & = \left( \frac{P_k}{P\ik}\right)^\theta C_k
      \numberthis \label{FOCCik}
\end{align*}


FOC $[C_k]$:

\begin{align*}
P_k & = P \left(C_k^{\frac{\rho-1}{\rho}}dk \right)^{\frac{1}{\rho-1}} C_k^{\frac{-1}{\rho}} 
\\
\Rightarrow
C_k & = \left( \frac{P}{P_k}\right)^\rho C 
    \numberthis \label{FOCCk} 
\end{align*}


We can plug these first order conditions into the original given definitions for $C$ and $C_k$:

\begin{align*}
C_k & = \left( \sum_{i=1}^{N_k} C_{ik}^{\frac{\theta-1}{\theta}}\right)^{\frac{\theta}{\theta-1}}
\\
& = \fix{XXXXX}
\\
P_k & = \left(\sumi P\ik ^{1-\theta}\right)
        ^{\frac{1}{\theta-1}}
        \tag{*} 
        % sectoral price index as a function of 
        % producer's prices, Pik. 
\end{align*}


\begin{align*}
C = \left( \int C_k^{\frac{\rho-1}{\rho}}dk\right)^{\frac{\rho}{\rho-1}}
& = \left[ \int \left( \left[ 
    \frac{P}{P_k}\right]^\rho C \right)^{\frac{\rho-1}
    {\rho}} dk \right]^{\frac{\rho}{\rho-1}}
\\
& = \left[C^{\frac{\rho-1}{\rho}} P^{\rho-1} 
    \left( \int P_k^{1-\rho} dk \right)
    \right]^{\frac{\rho}{\rho-1}} 
\\
P & = \left(\int P_k^{1-\rho}dk\right)^{\frac{1}{\rho-1}} 
        \tag{**} 
        % aggregate price index
\end{align*}

This gives us the sectoral price index and the aggregate price index as a function of producer's prices $P\ik$.
    \footnote{Note that we can plug (*) into (**) to 
    get 
    \[ P = \left(\int\left(\sumi P\ik ^{1-\theta}
    \right)^{\frac{1-\rho}{\theta-1}} 
    dk\right)^{\frac{1}{\rho-1}}
    \] 
    \fix{check this}}
\\
\\    
Now to get consumption as a function of producers' prices, we use first order conditions \eqref{FOCCik} and \eqref{FOCCk}




\end{document}
