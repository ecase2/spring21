% $•$% !TeX root = macro_PS6_case.tex

\documentclass[]{article}
    \setlength{\parindent}{0em}
\usepackage{amssymb}
\usepackage{amsmath}
\usepackage{enumitem}
\usepackage{bm}
\usepackage{cases}
\usepackage{changepage}\usepackage{amsmath}
\usepackage{amsfonts}
\usepackage{centernot}
\usepackage{graphicx}
\usepackage{float}
\usepackage{hyperref} % to restart section numbers for different parts
\usepackage{physics} % to get partial derivatives 
\usepackage{xcolor}
\usepackage[a4paper, total={6.5in, 8.5in}]{geometry}
\usepackage{listings} % to input code
       \lstset{backgroundcolor = \color{white},
               % frame = single,
               %keywordstyle=\color{blue}
               }

%opening
\title{Econ 713 Quarter 1: Problem set 1 }
\author{Emily Case}

% command for blank footnote 
\newcommand\blfootnote[1]{%
	\begingroup
	\renewcommand\thefootnote{}\footnote{#1}%
	\addtocounter{footnote}{-1}%
	\endgroup
}

\newcount\colveccount
\newcommand*\colvec[1]{
	\global\colveccount#1
	\begin{pmatrix}
		\colvecnext
	}
	\def\colvecnext#1{
		#1
		\global\advance\colveccount-1
		\ifnum\colveccount>0
		\\
		\expandafter\colvecnext
		\else
	\end{pmatrix}
	\fi
}

\newcommand{\R}{\mathbb{R}}
\newcommand{\co}[2]{c_{#1}^{#2}} % makes subscript and superscript easier for consumption
\newcommand{\spa}{\text{ }}
\newcommand{\lnl}{\ell_n} % log likelihood function
\newcommand{\bxn}{\bar{X}_n}
\newcommand{\lag}{\mathcal{L}}
\newcommand{\sumin}{\sum\limits_{i=1}^n} % generic sumation 1 to n
\newcommand{\sumti}{\sum\limits_{t=1}^\infty} % sum - infinite horizon
\newcommand{\pin}{\Pi_{i=1}^n}
\newcommand{\E}{\mathbb{E}}
\newcommand{\argmax}{\text{arg}\max}
\newcommand{\fix} [1] {\textbf{\textcolor{blue}{#1}}} % use this to more clearly note in problem set pdf where things need to be changed. 
% end preamble

\renewcommand*{\thesection}{\arabic{section}}

\begin{document}
	
	\maketitle
	
	\blfootnote{I worked on this Problem set with Sarah Bass, Michael Nattinger, Alex von Hafften, and Danny Edgel.} 


%%%%%%%%%%%%%%%%%%%%%%%%%%%%%%%%%%%%%%%%%%%%%%%%%%%%%%%%%%



\section{Consider the following non-transferable utility matching problem of three men and three women.  Unmatched payoff is zero for both men and women. Does the Gale-Shapley algorithm yield the same outcome if we let women propose to men instead of men propose to women?}

\begin{center}
\begin{tabular}{|c|c|c|c|}
\hline 
 & W1 & W2 & W3 \\ 
\hline 
M1 & 10,5 & 8,3 & 6,12 \\ 
\hline 
M2 & 4,10 & 5,2 & 3,20 \\  
\hline 
M3 & 6,15 & 7,1 & 8,16 \\ 
\hline 
\end{tabular}
\end{center}


\textbf{Men would make the following proposals:} 
\begin{center}
\begin{tabular}{|c|c|c|c|}
\hline 
 & W1 & W2 & W3 \\ 
\hline 
M1 & \textbf{10,5} & 8,3 & 6,12 \\ 
\hline 
M2 & 4,10 & \textbf{5,2} & 3,20 \\ 
\hline 
M3 & 6,15 & 7,1 & \textbf{8,16} \\ 
\hline 
\end{tabular}
\end{center}
Since every woman is proposed to, they will all accept (because they are better off than being single), and the stable matchings are the bolded ones. It takes one round, and is male-optimal/female-pessimal. 
\\\\

\textbf{Women would make the following proposals:}
\begin{center}
\begin{tabular}{|c|c|c|c|}
\hline 
 & W1 & W2 & W3 \\ 
\hline 
M1 & 10,5 & \textbf{8,3} & 6,12 \\ 
\hline 
M2 & 4,10 & 5,2 & \textbf{3,20} \\  
\hline 
M3 & \textbf{6,15} & 7,1 & 8,16 \\ 
\hline 
\end{tabular}
\end{center}
Again, each man has one proposal and so all will accept. Again it takes one round, but it is female-optimal/male-pessimal. 
\\
\\
\textbf{What happens if now we switch utilities for women 1 to be (10,5,15) and for women3 to be (12,16,20)?}
 \begin{center}
\begin{tabular}{|c|c|c|c|}
\hline 
 & W1 & W2 & W3 \\ 
\hline 
M1 & 10,10 & \textbf{8,3} & 6,12 \\ 
\hline 
M2 & 4,5 & 5,2 & 3,16 \\  
\hline 
M3 & \textbf{6,15} & 7,1 & \textbf{8,20} \\ 
\hline 
\end{tabular}
\end{center}
Men's proposals stay the same. Women's proposals in the first round are again bolded, and M3 has two women to choose from. He selects his preferred W3, leaving W1 to make her second proposal. Now, assuming all agents have perfect information, M1 knows that he is W1's second choice, so he will \textit{reject} his first proposal. \\
\\
For the second and final round, matched couples are in red, and new proposals are in bold. W1 proposes to her second best, as M1 hoped. W2 also has to make a proposal to M2\footnote{Her second preferred option is already taken!}:
\begin{center}
\begin{tabular}{|c|c|c|c|}
\hline 
 & W1 & W2 & W3 \\ 
\hline 
M1 & \textbf{10,10} & 8,3 & 6,12 \\ 
\hline 
M2 & 4,5 & \textbf{5,2} & 3,16 \\  
\hline 
M3 & 6,15 & 7,1 & \textbf{\textcolor{red}{8,20}} \\ 
\hline 
\end{tabular}
\end{center}
Note that now we have the same stable matching whether men or women propose, so it is unique. 



\section{(Econ 711 - Fall 2010 Q.1) Consider a matching market with two distinct “sides,” metaphorically  called  “men”  and  “women,”  but  perhaps  better  thought  of  as  a professional  partnership, likes specialist  neuro-surgeons  and  interns (one-on-one matches). } All benefits of the match are as given below:

\begin{center}
\begin{tabular}{|c|c|c|c|}
\hline 
      & M1 & M2 & M3 \\ 
\hline 
W1 & 1,2 & 4,3 & 3,2 \\ 
\hline 
W2 & 1,3 & 2,4 & 3,2 \\ 
\hline 
W3 & 2,2 & 2,2 & 4,4 \\ 
\hline 
\end{tabular} 
\end{center}

Unmatched individuals earn nothing. 

\begin{enumerate}[label = (\alph*)]

%%% (a) %%% 
\item \emph{Assume that wages are not negotiable, and thus no side transfers are possible. Find all stable matchings. Carefully justify your answer.} \\
\\

\textbf{Men would make the following bolded first proposals:}
\begin{center}
\begin{tabular}{|c|c|c|c|}
\hline 
  & M1 & M2 & M3 \\ 
\hline 
W1 & 1,2 & 4,3 & 3,2 \\ 
\hline 
W2 & \textbf{1,3} & \textbf{2,4} & 3,2 \\ 
\hline 
W3 & 2,2 & 2,2 & \textbf{4,4} \\ 
\hline 
\end{tabular} 
$\longrightarrow$
\begin{tabular}{|c|c|c|c|}
\hline 
  & M1 & M2 & M3 \\ 
\hline 
W1 & \textbf{\textcolor{red}{1,2}} & 4,3 & 3,2 \\ 
\hline 
W2 & 1,3 & \textbf{\textcolor{red}{2,4}} & 3,2 \\ 
\hline 
W3 & 2,2 & 2,2 & \textbf{\textcolor{red}{4,4}} \\ 
\hline 
\end{tabular} 
\end{center}

In this round, W3's favorite man has proposed to her, so she absolutely accepts. W2 has two proposals, and selects M2. W1 is lonely in the first round. \\
\\
In the second and final round (shown on the above right), M1 needs to make a new proposal, and his only option is W1, who will accept. We get one stable matching, which is male-optimal: \\
\\

\textbf{Women would make the following bolded first proposals:}

\begin{center}
\begin{tabular}{|c|c|c|c|}
\hline 
               & M1 & M2 & M3 \\ 
\hline 
W1 & 1,2 & \textbf{4,3} & 3,2 \\ 
\hline 
W2 & 1,3 & 2,4 & \textbf{3,2} \\ 
\hline 
W3 & 2,2 & 2,2 & \textbf{4,4} \\ 
\hline 
\end{tabular} 
$\longrightarrow$
\begin{tabular}{|c|c|c|c|}
\hline 
               & M1 & M2 & M3 \\ 
\hline 
W1 & 1,2 & \textbf{\textcolor{red}{4,3}} & 3,2 \\ 
\hline 
W2 & \textbf{\textcolor{red}{1,3}} & 2,4 & 3,2 \\ 
\hline 
W3 & 2,2 & 2,2 & \textbf{\textcolor{red}{4,4}} \\ 
\hline 
\end{tabular} 
\end{center}
M3 selects W3 of his options, leaving W2 single until next round. M2 accepts his offer because it's his best option in general. Final matchings are in red on the rightside table.\\
\\

This is a female-optimal stable matching that took two rounds. 
\\\\
\fix{Should get the same thing. change my logic}


%%% (b) %%% 
\item \textit{From now on,  assume side transfers are possible.  Let payoffs be the sum of transfers.  Find the efficient matching.} \\
The new payoffs are:

\begin{center}
\begin{tabular}{|c|c|c|c|}
\hline 
      & M1 & M2 & M3 \\ 
\hline 
W1 & 3 & \textbf{7} & 5 \\ 
\hline 
W2 & \textbf{4} & 6 & 5 \\ 
\hline 
W3 & 4 & 4 & \textbf{8} \\ 
\hline 
\end{tabular} 
\end{center}
and the most efficient matching is bolded, with a total payoff of 19.

%%% (c) %%% 
\item \textit{Find with proof the minimum wage for the type 2 man.}\footnote{Hint: Let the wages of women be $w_i$ and wages of men be $v_i$.}

\end{enumerate}



\section{Assume types are drawn uniformly from $[0,1]$.  When a type $x $ matches with a type $y$, type $x$ gets payoff $y+axy$, and the payoff to $y$ matching with $ x$ is symmetrically $x+axy$.  Assume$-1< a <0$.}

\begin{enumerate}[label = (\alph*)]

%%% (a) %%% 
\item If there are nontransferable payoffs, then:
\begin{itemize}
    \item Everyone who is type x, regardless of their ranking, prefers the type y who draws value 1 ($1_y$) in order to maximize their payoffs. 
    \item Symmetrically, all of type y will prefer the x with value 1 ($1_x$).  
\end{itemize}
Because of symmetry the matching will be the same no matter what, but consider that x makes the proposals. All x propose to $1_y$, who will accept $1_x$. Then remaining x all propose to the second highest y, who accepts the second highest x, and so on. We get stable matchings $(1_x,1_y),\; ... ,\; (0_x,0_y)$. In other words, x and y agents of the same ranking in the uniform distribution are matched. Mathematically we can say that the matching function for $x$ is:
\[x(y) = y\]


%%% (b) %%% 
\item Set up a matching market, with wage $w(x)$ for each type x.  What wage can decentralize this market?  (Can any other wage work?)\\
\\
Match makers in this economy profit $f(x,y) -w(x)-w(y)$, where $f(x,y) = x+y+2axy$. Note that the second derivative of the output is $f_{xy} = 2a<0$, so the optimum choice here is $y = 1-x$. Profit becomes
\[\pi(x,y)  = x+y+2axy-w(x)-w(y)\]
FOC[x]:
\[1+2ay-w'(x) = 1+2a(1-x)-w'(x) = 0\Rightarrow 
w'(x) = 1+2a(1-x)\]
Which we now integrate and get $w(x) = x+2ax-ax^2 + c$ for some constant $c$. We can impose the condition of free entry and exit of matchmakers to get 
\begin{align*}
0 & = x+y+2axy-w(x)-w(y) 
\\
0 & = 1+2ax(1-x) -w(x) -w(1-x)
\\
0 & = 1 +2ax(1-x) -[x+2ax-ax^2 + c]
    - [1-x+2a(1-x)-a(1-x)^2 + c]
    \\
2c & = 2ax-2ax^2 +2ax-ax^2-2a+2ax +a-a2x+ax^2
\\
c & = a/2
\end{align*}
So the wage is $w(x) = x+2ax-ax^2 + a/2$. 


% end question 2
\end{enumerate}


\section{Double Auctions - The Borrowers: There are 30 students numbered 1,2,3, ...,30. Even students are lenders and odd students are borrowers. The lenders each have 1000 to lend, and the return available to any lender is 3\% plus 0.01\% (known as a basis point) times twice his student number.  The borrower has a return on a project equal to 3\% plus 0.01\% times his student number.  So a borrower borrows if he can get an interest rate below his project's return, and a lender lends if he can get an interest rate above the return he has available to him.  What are all possible market clearing interest rates, and numbers of transactions?}
Lenders: student 2 gets a return of 3.04\%, 4 gets 3.08\%, 6 gets 3.12\%, and so on. Borrowers get project returns: student 1 gets 3.01\%, student 3 gets 3.03\%, and so on. \\
\\
Student 1's project return is lower than any possible return to the lenders, and so no one will match with student 1. Student 30 has the highest rate, 3.60\%, yet the highest any of the borrowers are willing to go is 3.29\%, and so no one will want to borrow from student 30. In fact, no one will want to borrow from even numbered students 16,18,...30.  That leaves 7 lenders from which to borrow. \\
\\
In order for the market to clear, the interest rate must be at least 3.15 to 3.16\%. If it is less than 3.15\%, then there are 8 who wish to borrow, but only 3 willing to lend. If it is above 3.16\%, then there are at most 7 willing to borrow, but 4 willing to lend. \\
\fix{I must be missing something.}


\end{document}
