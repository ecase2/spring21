% $•$% !TeX root = macro_PS6_case.tex

\documentclass[12pt,oneside,reqno]{amsart}
    % \documentclass[]{article}
\usepackage{amssymb}
\usepackage{amsmath}
\usepackage{enumitem}
\usepackage{bm}
\usepackage{cases}
\usepackage{changepage}
\usepackage{amsfonts}
\usepackage{centernot}
\usepackage{graphicx}
\usepackage{float}
\usepackage{hyperref} % to restart section numbers for different parts
\usepackage{physics} % to get partial derivatives 
\usepackage{xcolor}
\usepackage[a4paper, total={6in, 8in}]{geometry}
\usepackage{listings} % to input code
       \lstset{backgroundcolor = \color{white},
               % frame = single,
               %keywordstyle=\color{blue}
               }
\usepackage{etoolbox}
% \patchcmd{<cmd>}{<search>}{<replace>}{<success>}{<failure>}
\patchcmd{\section}{\centering}{}{}{}
    % made section title not centered 
\let\flushleftsection\section% Copy updated non-centered definition of \section
\newcommand{\sectionscenter}{\let\section\centeredsection}% Switch to centered \section
\newcommand{\sectionsleft}{\let\section\flushleftsection}% Switch to flush left \section

\setlength{\parindent}{0em} % no automatic indents

%opening
\title{Econ 710 Quarter 1: Problem set 2 }
\author{Emily Case}

% command for blank footnote 
\newcommand\blfootnote[1]{%
	\begingroup
	\renewcommand\thefootnote{}\footnote{#1}%
	\addtocounter{footnote}{-1}%
	\endgroup
}

\newcount\colveccount
\newcommand*\colvec[1]{
	\global\colveccount#1
	\begin{pmatrix}
		\colvecnext
	}
	\def\colvecnext#1{
		#1
		\global\advance\colveccount-1
		\ifnum\colveccount>0
		\\
		\expandafter\colvecnext
		\else
	\end{pmatrix}
	\fi
}

\newcommand{\R}{\mathbb{R}}
\newcommand{\E}{\mathbb{E}}
\newcommand{\co}[2]{c_{#1}^{#2}} % makes subscript and superscript easier for consumption
\newcommand{\spa}{\text{ }}
\newcommand{\lnl}{\ell_n} % log likelihood function
\newcommand{\bxn}{\bar{X}_n}
\newcommand{\lag}{\mathcal{L}}
\newcommand{\sumin}{\sum\limits_{i=1}^n} % generic sumation 1 to n
\newcommand{\sumti}{\sum\limits_{t=0}^\infty} % sum - infinite horizon
\newcommand{\pin}{\Pi_{i=1}^n}
\newcommand{\argmax}{\text{arg}\max}
\newcommand{\fix} [1] {\textbf{\textcolor{blue}{#1}}} % use this to more clearly note in problem set pdf where things need to be changed. 
% end preamble

\renewcommand*{\thesection}{\arabic{section}}
\newcommand\numberthis{\addtocounter{equation}{1}\tag{\theequation}} % for equation numbering within align*

\begin{document}
	
	\maketitle
	
	\blfootnote{I worked on this Problem set with Sarah Bass, Michael Nattinger, Alex von Hafften, and Danny Edgel.} 


%%%%%%%%%%%%%%%%%%%%%%%%%%%%%%%%%%%%%%%%%%%%%%%%%%%%%%%%%%
\section*{Exercise 1}

Suppose $(Y,X,Z)'$ is a vector of random variables such that 
\begin{align*}
Y  & = \beta_0 +X\beta_1 +U, 
& \E[U|Z]  = 2
\end{align*}
where $Cov(Z,X) \neq 0$ and $\E[Y^2 +X^2+Z^2]<\infty$. Additionally, let $\{(Y_i,X_i,Z_i)'\}_{i=1}^n$ be a random sample from the model with $\widehat{Cov}(Z,X) \neq 0$. 
\begin{align*}
\hat{\beta}_1^{IV} & = \frac{\widehat{Cov}(Z,Y)}{\widehat{Cov}(Z,X)} 
= \frac{\frac{1}{n} \sumin(Z_i-\bar{Z}_n)(Y_i-\bar{Y}_n)} {\frac{1}{n}\sumin (Z_i-\bar{Z}_n)(X_i-\bar{X}_n)}
\\
\hat{\beta}_0^{IV} & =\bar{Y}-\bar{X}\hat{\beta}_1^{IV}
\end{align*}


% problems %
\begin{enumerate}[label = (\roman*)]

%%% (i) %%%
\item \textit{Does $\hat{\beta}_1^{IV} \rightarrow^p \beta_1$?} \\
Note that by LLN and CMT, 
\[\widehat{Cov}(Z,Y) = \frac{1}{n} \sumin(Z_i-\bar{Z}_n)(Y_i-\bar{Y}_n) \rightarrow^p \E[(Z_i-\bar{Z}_n)(Y_i-\bar{Y}_n) = Cov(Z,Y)\]  
and by similar logic $\widehat{Cov}(Z,X) \rightarrow^p Cov(Z,X)$. Then, by CMT, 
\begin{align*}
\hat{\beta}^{IV}_1 & \rightarrow^p \frac{Cov(Z,Y)}{Cov(Z,X)} 
\\
& = \frac{Cov(Z, \beta_0 +X\beta_1 +U)}
    {Cov(Z,X)} 
    \\
& = \frac{\beta_1Cov(Z,X) +Cov(Z,U)}
    {Cov(Z,X)} 
    \\
& = \beta_1
\end{align*}
Since $Cov(Z,U) = 0$.
\footnote{ \fix{check this}
\begin{align*}
Cov(Z,U) & = \E[(Z-\E[Z])(U-\E[U])] 
\\
& = \E[ZU] -\E[\E[Z]U] -\E[Z\E[U]] +\E[\E[Z]\E[U]]
\\
& = \E[ZU] -\E[Z]\E[U] 
\\
& = \E[Z\E[U|Z]] - E[Z]\E[\E[U|Z]] \\
& = 2\E[Z] - 2E[Z]
\end{align*}
}
\\


%%% (ii) %%%
\item \textit{Does $\hat{\beta}_0^{IV} \rightarrow^p \beta_0$?}
\begin{align*}
\hat{\beta}_0^{IV} 
& = \bar{Y}-\bar{X}\hat{\beta}_1^{IV} 
\\
& \rightarrow^2 \bar{Y}-\bar{X}\beta_1
\end{align*}

% end exercise 1 % 
\end{enumerate}



\section*{Exercise 2}

Consider the simultaneous model 

\begin{align*}
& Y = \beta_0+X\beta_1 +U, &\E[U|Z]= 0 
\\
& X = \pi_0 +Z\pi_1+V, & \E[V|Z] = 0
\end{align*}

where $\E[Y^2 +X^2+Z^2]<\infty$. Additionally, let $\{(Y_i,X_i,Z_i)'\}_{i=1}^n$ be a random sample from the model with $\frac{1}{n}\sumin (Z_i-\bar{Z}_n)^2>0$. 


% problems % 
\begin{enumerate}[label = (\roman*)]

%%% (i) %%%
\item \textit{Under what conditions (on $\beta_0,\;\beta_1,\;\pi_0,\;\pi_1)$ is $Z$ a valid instrument for X?}


%%% (ii) %%%
\item \textit{Show that}
\begin{align*}
& Y = \gamma_0 +Z\gamma_1 +\epsilon, 
& \E[\epsilon|Z] = 0
\end{align*}
\textit{where $\gamma_0,\;\gamma_1$ and $\epsilon$ are some functions of $\beta_0,\;\beta_1,\;\pi_0,\;\pi_1,\; U,$ and $V$. In particular show that $\gamma_1 = \pi_1\beta_1$.}


%%% (iii) %%% 


%%% (iv) %%%


%%% (v) %%%


% end exercise 2 % 
\end{enumerate}



\section*{Exercise 3}

The paper “Children and Their Parents’ Labor Supply: Evidence fromExogenous Variation in Family Size” by J. Angrist and W. Evans (AE98) considerslabor supply responses to the number of children in the household. 

They consider models of the form
\[Y = \beta_0 +X_1\beta_1+X'_2\beta_2 +U \]
where $Y$ is some measure of the parents’ labor supply, $X_1$ is a binary variable indicating “more than 2 children in the household”, and $X_2$ is a vector of (assumed) exogenous variables that control for race, age, and whether any of the children is a boy. For the next two questions we will focus on the case where $Y$ is a binary variable indicating whether the mother worked during the year.


\begin{enumerate}[label = (\roman*)]

%%% (i) %%%
\item \textit{Provide a causal interpretation of $\beta_1$.}


%%% (ii) %%%
\item \textit{Discuss why or why not you think that $X_1$ could be endogenous. If you think it is, discuss the direction of the (conditional) bias in OLS relative to the causal parameter.}


%%% (iii) %%%
\item \textit{Repeat the previous two questions when Y is a binary variable indicating whether the husband worked during the year.}


%%% (iv) %%%
\item \textit{Discuss why or why not you think that the binary variable $Z_1$ which indicates whether the two first children are of the same sex is a valid instrument for $X_1$.}


%%% (v) %%% 
\item \textit{Estimate the reduced form regression of $X_1$ on $Z_1$ and $X_2$, do the results suggest that $Z_1$ is relevant?}


%%% (vi) %%%
\item \textit{(Attempt to) replicate the first three rows of Table 7 columns 1, 2, 5, 7, and 8 in AE98. Interpret the empirical results in relation to your discussion of the previous questions.}

% end exercise 3 % 
\end{enumerate}




\end{document}
