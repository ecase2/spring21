% $•$% !TeX root = macro_PS6_case.tex

\documentclass[]{article}
\usepackage{amssymb}
\usepackage{amsmath}
\usepackage{enumitem}
\usepackage{bm}
\usepackage{cases}
\usepackage{changepage}\usepackage{amsmath}
\usepackage{amsfonts}
\usepackage{centernot}
\usepackage{graphicx}
\usepackage{float}
\usepackage{hyperref} % to restart section numbers for different parts
\usepackage{physics} % to get partial derivatives 
\usepackage{xcolor}
\usepackage[a4paper, total={6.5in, 8.5in}]{geometry}
\usepackage{listings} % to input code
       \lstset{backgroundcolor = \color{white},
               % frame = single,
               %keywordstyle=\color{blue}
               }

%opening
\title{Econ 710 Quarter 1: Problem set 1 }
\author{Emily Case}

% command for blank footnote 
\newcommand\blfootnote[1]{%
	\begingroup
	\renewcommand\thefootnote{}\footnote{#1}%
	\addtocounter{footnote}{-1}%
	\endgroup
}

\newcount\colveccount
\newcommand*\colvec[1]{
	\global\colveccount#1
	\begin{pmatrix}
		\colvecnext
	}
	\def\colvecnext#1{
		#1
		\global\advance\colveccount-1
		\ifnum\colveccount>0
		\\
		\expandafter\colvecnext
		\else
	\end{pmatrix}
	\fi
}

\newcommand{\R}{\mathbb{R}}
\newcommand{\co}[2]{c_{#1}^{#2}} % makes subscript and superscript easier for consumption
\newcommand{\spa}{\text{ }}
\newcommand{\lnl}{\ell_n} % log likelihood function
\newcommand{\bxn}{\bar{X}_n}
\newcommand{\lag}{\mathcal{L}}
\newcommand{\sumin}{\sum\limits_{i=1}^n} % generic sumation 1 to n
\newcommand{\sumti}{\sum\limits_{t=1}^\infty} % sum - infinite horizon
\newcommand{\pin}{\Pi_{i=1}^n}
\newcommand{\E}{\mathbb{E}}
\newcommand{\argmax}{\text{arg}\max}
\newcommand{\fix} [1] {\textbf{\textcolor{blue}{#1}}} % use this to more clearly note in problem set pdf where things need to be changed. 
% end preamble

\renewcommand*{\thesection}{\arabic{section}}

\begin{document}
	
	\maketitle
	
	\blfootnote{I worked on this Problem set with Sarah Bass, Michael Nattinger, Alex von Hafften, and Danny Edgel.} 


%%%%%%%%%%%%%%%%%%%%%%%%%%%%%%%%%%%%%%%%%%%%%%%%%%%%%%%%%%



\section*{Exercise 1}
Suppose $(Y,X')'$ is a random vector with \[Y = X'\beta_0\cdot U\] where $\mathbb{E}[U|X] = 1,\;  \mathbb{E} [XX']$ is invertible, and $\mathbb{E} [Y^2 +||X||^2] < \infty $. Furthermore, suppose that $\{Y_i,X'_i\}_{i=1}^\infty$ is a random sample from the distribution of $(Y,X')'$, where $\frac{1}{n} \sumin X_iX'_i$ is invertible and let $\hat{\beta}$ be the OLS estimator:
\[\hat{\beta} = \left(\frac{1}{n} \sumin X_iX'_i\right)^{-1} \frac{1}{n} \sumin X_iY_i\]

\begin{enumerate}[label = (\roman*)]

%%% (i) %%%
\item \textbf{Interpret the entries of $\beta_0$ in this model.} Here, when we take the derivative with respect to $X$, we will get a value/slope that is affected by the unobservables as well. Therefore $\beta_0$ is the value of the slope if it were unaffected by unobservable factors. \fix{there's probably a better way to say this}


%%% (ii) %%%  done 
\item \textbf{Show that $Y= X'\beta_0+\tilde{U}$ where $\mathbb{E} [\tilde{U}|X] = 0$.} \\
Notice that $\tilde{U} = Y - X'\beta_0 = X'\beta_0\cdot U - X'\beta_0 = X'\beta_0(U-1)$. So, when $\tilde{U} = X'\beta_0(U-1)$, the statement is true. Also, 

\begin{align*}
\E[\tilde{U} |X] 
& = \E[X'\beta_0(U-1) |X] \\
& = X'\beta_0\E[U|X] - X'\beta_0 \\
& = X'\beta_0 - X'\beta_0 = 0 
\end{align*}


%%% (iii) %%%
\item \textbf{Show that   $\E[X(Y - X'\beta)] = 0$ if and only if $\beta = \beta_0$ and use this to derive OLS as a method of moments estimator.} \\
\\
First, let $\beta = \beta_0$. Then 
\begin{align*}
\E[X(Y - X'\beta)] & = \E[X(Y - X'\beta_0)] \\
& = \E[XY - XX'\beta_0] \\
& = \E[XX'\beta_0\cdot U] - \E[XX'\beta_0]  \\
& = \E[\E[XX'\beta_0\cdot U|X]] - \E[XX'\beta_0] \\
& = \E[XX'\beta_0\E[U|X]] - \E[XX'\beta_0]\\
& = \E[XX'\beta_0] - \E[XX'\beta_0] \\
& = 0 
\end{align*}
Now suppose that $\E[X(Y - X'\beta)] = 0$. Then
\begin{align*}
0 & = \E[XY] - \E[XX'\beta] \\
& =  \E[X(X'\beta_0 +\tilde{ U})] - \E[XX'\beta] \\
\E[XX'\beta] & = \E[XX'\beta_0] + \E[X\tilde{U}] \\
\E[XX'\beta] & = \E[XX'\beta_0] + \E[\E[X\tilde{U}|X]] \\
\E[XX'\beta] & = \E[XX'\beta_0] \\
\iff \beta & = \beta_0
\end{align*}
\fix{ unsure about moments estimator? } 


%%% (iv) %%% done 
\item \textbf{Show that the OLS estimator is conditionally unbiased.}\\
\\
We need to show that $\E[\hat{\beta}|X_1,...,X_n] = \beta_0$:
\begin{align*}
\E[\hat{\beta}|X_1,...,X_n] 
& = \E\left[\left(\frac{1}{n} \sumin X_iX'_i\right)^{-1} \frac{1}{n} \sumin X_iY_i\bigg|X_1,...,X_n\right]
\\
& = \left(\frac{1}{n} \sumin X_iX'_i\right)^{-1} 
    \frac{1}{n}\sumin X_i \E[Y_i|X_i]
\\
& = \left(\frac{1}{n} \sumin X_iX'_i\right)^{-1} 
    \frac{1}{n}\sumin X_i 
    \E[X'_i\beta_0 + \tilde{U}|X_i]
\\
& = \left(\frac{1}{n} \sumin X_iX'_i\right)^{-1} 
    \frac{1}{n}\sumin X_i (
    X'_i\beta_0 + \E[\tilde{U}|X_i])
\\
& = \left(\frac{1}{n} \sumin X_iX'_i\right)^{-1} 
    \frac{1}{n}\sumin X_i X'_i\beta_0
\\
& = \left(\frac{1}{n} \sumin X_iX'_i\right)^{-1} 
    \left(\frac{1}{n}\sumin X_i X'_i\right)\beta_0
\\
& = \beta_0
\end{align*}


%%% (v) %%% 
\item \textbf{Show that the OLS estimator is consistent.}
By the law of large numbers, $ \frac{1}{n} \sumin X_iX'_i \rightarrow^p \E[XX']$ and $\frac{1}{n} \sumin X_iY_i \rightarrow^p \E[XY]$. We know that $\E[XX']$ is invertible. Now, by the continuous mapping theorem, we get that 
\begin{align*}
\hat{\beta} & \rightarrow^p \E[XX']^{-1} \E[XY]
\\
& = \E[XX']^{-1} \E[X(X'\beta_0 +\tilde{U})]
\\
& = \E[XX']^{-1} \E[XX'\beta_0 +X\tilde{U})]
\\
& = \beta_0 + \E[X\tilde{U})]
\\
& = \beta_0
\end{align*}


\end{enumerate}



\section*{Exercise 2} 
Let $X$ be a random variable with $\E[X^4]<\infty$ and $\E[X^2] > 0$. Furthermore, let $\{X_i\}^n_{i=1}$ be a random sample from the distribution of $X$.

\begin{enumerate}[label = (\roman*)]


%%% (i) %%% done 
\item \textbf{Which of the following four statistics can you use the law of large numbers and continuous mapping theorem to show convergence in probability as $n \rightarrow \infty$?}

\begin{enumerate}[label = (\alph*)]

%%% (a) %%%
\item  $\frac{1}{n}\sumin X_i^3$ \\
Since the fourth moment is finite, the third moment is also finite. Also, $\E[-X^3] <\infty$, so $\E[|X^3|] <\infty$. Now we can apply the law of large numbers:
\[\frac{1}{n}\sumin X_i^3 \rightarrow^p \E[X_i^3] \]

%%% (b) %%%
\item $\max_{1\le i \le n} X_i$ \\
We cannot use CMT here. 

%%% (c) %%%
\item $ \frac{\sumin X_i^3}{\sumin X_i^2}$ \\
LLN\footnote{Using similar logic to part a.} gives us that $\sumin X_i^3 \rightarrow^p \E[X^3]$ and $ \sumin X_i^2 \rightarrow^p$. Then using CMT, 
\[\frac{\sumin X_i^3}{\sumin X_i^2} \rightarrow^p \frac{\E[X^3]}{\E[X^2]}\]

%%% (d) %%%
\item $ 1 \bigg\{\frac{1}{n}\sumin X_i > 0\bigg\}$ \\
We must have that $\E[|X|] <\infty$, so $\frac{1}{n}\sumin X_i \rightarrow^p \E[X]$. So, by CMT, if $\E[X]>0$ it converges in probability to 1. If $\E[X]<0$ it converges in probability to 0. The indicator function is not continuous when $\E[X]=0$, so we can't use CMT in that case. 
\\\\
\fix{I want to talk through this one and maybe also just the general assumptions for these theorems and why they are necessary}

\end{enumerate}


%%% (ii) %%%
\item \textbf{For which of the following three statistics can you use the central limit theorem and continuous mapping to show convergence in distribution as $n \rightarrow \infty$?}

\begin{enumerate}[label = (\alph*)]

%%% (a) %%%
\item 

%%% (b) %%%
\item

%%% (c) %%%
\item 

\end{enumerate}


%%% (iii) %%% 
\item \textbf{Show that $\max_{1\le i \le n} X_i \rightarrow^p 1$ if $X \sim uniform(0,1)$}


%%% (iv) %%% 
\item \textbf{Show that Pr$(\max_{1\le i \le n} X_i >M \rightarrow 1$ for any $M\ge 0$ if $X \sim exponential(1)$}

\end{enumerate}



\section*{Exercise 3}
Suppose that $\{X_i\}_{i=1}^n $ is an i.i.d. sequence of N(0,1) random variables. Let W be independent of $\{X_i\}_{i=1}^n $ with Pr$(W=1)= Pr(W=-1)=1/2$. Let $Y_i = X_iW$. 

\begin{enumerate}[label = (\roman*)]

%%% (i) %%%  done
\item Show that $\frac{1}{\sqrt{n}} \sumin X_i \rightarrow^d N(0,1)$ as $n \rightarrow \infty$.
First, note that $\E[X_i^2] = Var(X_i) = 1 < \infty$. This meets the conditions for Lindeberg-Levy:
\[\frac{1}{\sqrt{n}} \sumin X_i = \frac{1}{\sqrt{n}}\sumin(X_i - \E[X_1]) 
\rightarrow^d N(0,Var(X_1)) = N(0,1)\]

%%% (ii) %%%  done
\item Show that $\frac{1}{\sqrt{n}} \sumin Y_i \rightarrow^d N(0,1)$ as $n \rightarrow \infty$.
There are two cases, either $Y_i = X_i$ (W=1) in which the answer is the same as (i), or $Y_i = -X_i$ (W=-1). Normal distributions are symmetric, and so $Y_i = -X_i$ will have the same distribution. In either case, 
\[\frac{1}{\sqrt{n}} \sumin Y_i \rightarrow^d N(0,1)\]

%%% (iii) %%% done
\item Show that $Cov(X_i,Y_i) =0$.
Note that $\E[X_iW] = 0$.
\begin{align*}
Cov(X_i,Y_i) & = \E[(X_i - \E[X_i])(Y_i-\E[Y_i])]
\\
& = \E[(X_i - \E[X_i])(X_iW-\E[X_iW])]
\\
& = \E[(X_i - \E[X_i])(X_iW)]
\\
& = \E[ X_i^2W]
\\ 
& = \E[X_i^2]\E[W] 
\\
& = 1*0 = 0 
\end{align*}

%%% (iv) %%%
\item Does $V := \frac{1}{\sqrt{n}} \sumin (X_i,Y_i)'\rightarrow^d N(0,I_2)$ as $n \rightarrow \infty$?
\fix{come back to this}

%%% (v) %%%
\item How does this exercise relate to the Cramer-Wold device introduced in lecture 2?
\fix{having trouble drawing connection to the definition in the notes}

\end{enumerate}
 
\end{document}
