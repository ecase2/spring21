% $•$% !TeX root = macro_PS6_case.tex

\documentclass[]{article}
\usepackage{amssymb}
\usepackage{amsmath}
\usepackage{enumitem}
\usepackage{bm}
\usepackage{cases}
\usepackage{changepage}\usepackage{amsmath}
\usepackage{amsfonts}
\usepackage{centernot}
\usepackage{graphicx}
\usepackage{float}
\usepackage{hyperref} % to restart section numbers for different parts
\usepackage{physics} % to get partial derivatives 
\usepackage{xcolor}
\usepackage[a4paper, total={6.5in, 8.5in}]{geometry}
\usepackage{listings} % to input code
       \lstset{backgroundcolor = \color{white},
               % frame = single,
               %keywordstyle=\color{blue}
               }

%opening
\title{Econ 710 Quarter 1: Problem set 1 }
\author{Emily Case}

% command for blank footnote 
\newcommand\blfootnote[1]{%
	\begingroup
	\renewcommand\thefootnote{}\footnote{#1}%
	\addtocounter{footnote}{-1}%
	\endgroup
}

\newcount\colveccount
\newcommand*\colvec[1]{
	\global\colveccount#1
	\begin{pmatrix}
		\colvecnext
	}
	\def\colvecnext#1{
		#1
		\global\advance\colveccount-1
		\ifnum\colveccount>0
		\\
		\expandafter\colvecnext
		\else
	\end{pmatrix}
	\fi
}

\newcommand{\R}{\mathbb{R}}
\newcommand{\co}[2]{c_{#1}^{#2}} % makes subscript and superscript easier for consumption
\newcommand{\spa}{\text{ }}
\newcommand{\lnl}{\ell_n} % log likelihood function
\newcommand{\bxn}{\bar{X}_n}
\newcommand{\lag}{\mathcal{L}}
\newcommand{\sumin}{\sum\limits_{i=1}^n} % generic sumation 1 to n
\newcommand{\sumti}{\sum\limits_{t=1}^\infty} % sum - infinite horizon
\newcommand{\pin}{\Pi_{i=1}^n}
\newcommand{\E}{\mathbb{E}}
\newcommand{\argmax}{\text{arg}\max}
\newcommand{\fix} [1] {\textbf{\textcolor{blue}{#1}}} % use this to more clearly note in problem set pdf where things need to be changed. 
% end preamble

\renewcommand*{\thesection}{\arabic{section}}

\begin{document}
	
	\maketitle
	
	\blfootnote{I worked on this Problem set with Sarah Bass, Michael Nattinger, Alex von Hafften, and Danny Edgel.} 


%%%%%%%%%%%%%%%%%%%%%%%%%%%%%%%%%%%%%%%%%%%%%%%%%%%%%%%%%%
\section*{Exercise 1}
Suppose $(Y,X')'$ is a random vector with \[Y = X'\beta_0\cdot U\] where $\mathbb{E}[U|X] = 1,\;  \mathbb{E} [XX']$ is invertible, and $\mathbb{E} [Y^2 +||X||^2] < \infty $. Furthermore, suppose that $\{Y_i,X'_i\}_{i=1}^\infty$ is a random sample from the distribution of $(Y,X')'$, where $\frac{1}{n} \sumin X_iX'_i$ is invertible and let $\hat{\beta}$ be the OLS estimator:
\[\hat{\beta} = \left(\frac{1}{n} \sumin X_iX'_i\right)^{-1} \frac{1}{n} \sumin X_iY_i\]

\begin{enumerate}[label = (\roman*)]

%%% (i) %%%
\item \textbf{Interpret the entries of $\beta_0$ in this model.} Here, when we take the derivative with respect to $X$, we will get a value/slope that is affected by the unobservables as well. Therefore $\beta_0$ is the value of the slope if it were unaffected by unobservable factors. \fix{there's probably a better way to say this}


%%% (ii) %%%  done 
\item \textbf{Show that $Y= X'\beta_0+\tilde{U}$ where $\mathbb{E} [\tilde{U}|X] = 0$.} \\
Notice that $\tilde{U} = Y - X'\beta_0 = X'\beta_0\cdot U - X'\beta_0 = X'\beta_0(U-1)$. So, when $\tilde{U} = X'\beta_0(U-1)$, the statement is true. Also, 

\begin{align*}
\E[\tilde{U} |X] 
& = \E[X'\beta_0(U-1) |X] \\
& = X'\beta_0\E[U|X] - X'\beta_0 \\
& = X'\beta_0 - X'\beta_0 = 0 
\end{align*}


%%% (iii) %%%
\item \textbf{Show that   $\E[X(Y - X'\beta)] = 0$ if and only if $\beta = \beta_0$ and use this to derive OLS as a method of moments estimator.} \\
\\
First, let $\beta = \beta_0$. Then 
\begin{align*}
\E[X(Y - X'\beta)] & = \E[X(Y - X'\beta_0)] \\
& = \E[XY - XX'\beta_0] \\
& = \E[XX'\beta_0\cdot U] - \E[XX'\beta_0]  \\
& = \E[\E[XX'\beta_0\cdot U|X]] - \E[XX'\beta_0] \\
& = \E[XX'\beta_0\E[U|X]] - \E[XX'\beta_0]\\
& = \E[XX'\beta_0] - \E[XX'\beta_0] \\
& = 0 
\end{align*}
Now suppose that $\E[X(Y - X'\beta)] = 0$. Then
\begin{align*}
0 & = \E[XY] - \E[XX'\beta] \\
& =  \E[X(X'\beta_0 +\tilde{ U})] - \E[XX'\beta] \\
\E[XX'\beta] & = \E[XX'\beta_0] + \E[X\tilde{U}] \\
\E[XX'\beta] & = \E[XX'\beta_0] + \E[\E[X\tilde{U}|X]] \\
\E[XX'\beta] & = \E[XX'\beta_0] \\
\iff \beta & = \beta_0
\end{align*}
\fix{ unsure about moments estimator? } 


%%% (iv) %%% done 
\item \textbf{Show that the OLS estimator is conditionally unbiased.}\\
\\
We need to show that $\E[\hat{\beta}|X_1,...,X_n] = \beta_0$:
\begin{align*}
\E[\hat{\beta}|X_1,...,X_n] 
& = \E\left[\left(\frac{1}{n} \sumin X_iX'_i\right)^{-1} \frac{1}{n} \sumin X_iY_i\bigg|X_1,...,X_n\right]
\\
& = \left(\frac{1}{n} \sumin X_iX'_i\right)^{-1} 
    \frac{1}{n}\sumin X_i \E[Y_i|X_i]
\\
& = \left(\frac{1}{n} \sumin X_iX'_i\right)^{-1} 
    \frac{1}{n}\sumin X_i 
    \E[X'_i\beta_0 + \tilde{U}|X_i]
\\
& = \left(\frac{1}{n} \sumin X_iX'_i\right)^{-1} 
    \frac{1}{n}\sumin X_i (
    X'_i\beta_0 + \E[\tilde{U}|X_i])
\\
& = \left(\frac{1}{n} \sumin X_iX'_i\right)^{-1} 
    \frac{1}{n}\sumin X_i X'_i\beta_0
\\
& = \left(\frac{1}{n} \sumin X_iX'_i\right)^{-1} 
    \left(\frac{1}{n}\sumin X_i X'_i\right)\beta_0
\\
& = \beta_0
\end{align*}


%%% (v) %%% 
\item \textbf{Show that the OLS estimator is consistent.}
First note that 
\begin{align*}
\hat{\beta} & = \left(\frac{1}{n} \sumin X_iX'_i\right)^{-1} \frac{1}{n} \sumin X_iY_i\\
& = \left(\frac{1}{n} \sumin X_iX'_i\right)^{-1} \left(\frac{1}{n} \sumin X_iX'_i\beta_0 + X_i\tilde{U}_i\right) \\
& = \beta_0 + \left(\frac{1}{n} \sumin X_iX'_i\right)^{-1} \frac{1}{n} \sumin X_i\tilde{U}_i
\end{align*}
Which converges \fix{in distribution} to $\beta_0 + \E[X_iX'_i]^{-1} E[X_i\tilde{U}_i]$. Now, because $\mathbb{E} [\tilde{U}|X] = 0$,
\begin{align*}
\hat{\beta} 
& \rightarrow^p \beta_0 + \E[X_iX'_i]^{-1} E[X_i\tilde{U}_i] \\
& = \beta_0 
\end{align*}

\fix{need to polish this}

\end{enumerate}


\section*{Exercise 2} 
Let $X$ be a random variable with $\E[X^4]<\infty$ and $\E[X^2] > 0$. Furthermore, let $\{X_i\}^n_{i=1}$ be a random sample from the distribution of $X$.

\begin{enumerate}[label = (\roman*)]

\item \textbf{Which of the following four statistics can you use the law of large numbers and continuous mapping theorem to show convergence in probability as $n \rightarrow \infty$?}
\begin{enumerate}[label = (\alph*)]

\item  \[\frac{1}{n}\sumin X_i^3 \rightarrow^p \E[X_i^3]\]

\item \[\max_{1\le i \le n} X_i\] \fix{cannot use CMT?}

\item \[ \frac{\sumin X_i^3}{\sumin X_i^2}\]

\item \[ 1 \bigg\{\frac{1}{n}\sumin X_i > 0\bigg\}\]

\end{enumerate}

\item \textbf{For which of the following three statistics can you use the central limit theorem and continuous mapping to show convergence in distribution as $n \rightarrow \infty$?}

\begin{enumerate}[label = (\alph*)]
\item 

\item

\item 

\end{enumerate}


\end{enumerate}

\section*{Exercise 3}
\end{document}
